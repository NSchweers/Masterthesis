% Created 2016-03-16 Wed 18:29
\documentclass[a4paper]{article}
\usepackage[utf8]{inputenc}
\usepackage[T1]{fontenc}
\usepackage{fixltx2e}
\usepackage{graphicx}
\usepackage{longtable}
\usepackage{float}
\usepackage{wrapfig}
\usepackage{rotating}
\usepackage[normalem]{ulem}
\usepackage{amsmath}
\usepackage{textcomp}
\usepackage{marvosym}
\usepackage{wasysym}
\usepackage{amssymb}
\usepackage[hidelinks]{hyperref}
\tolerance=1000
\usepackage{minted}
\usepackage{todonotes}
\author{Nathanael Schweers}
\date{\today}
\title{Proof of Concept: An implementation of a reader for extensible syntaxes for Emacs Lisp}
\hypersetup{
 pdfauthor={Nathanael Schweers},
 pdftitle={Proof of Concept: An implementation of a reader for extensible syntaxes for Emacs Lisp},
 pdfkeywords={},
 pdfsubject={},
 pdfcreator={Emacs 25.0.92.1 (Org mode 8.3.4)}, 
 pdflang={English}}
\begin{document}

\maketitle
\tableofcontents

\begin{abstract}
This thesis discusses the design and implementation of a proof of concept for
an Emacs Lisp compatible reader for extensible syntaxes, in Emacs Lisp.  In Lisp
parlance, the reader is the part of the interpreter/compiler, which converts
the characters in the source code to data structures, which are then further
processed.  In an extensible reader, new syntactic constructs may be defined.
These new constructs take effect when specific characters are encountered in
the source code, and may interact with the existing reader to create arbitrary
data structures which are later either interpreted or compiled by the Lisp
environment.

Some other languages feature such capabilities, of which Common Lisp is most
widely known.  Some Scheme implementations also provide these or similar
possibilities, yet this is not part of the standard.  SRFI-10 provides a
specific character sequence with which all read macros must start.
\todo{Provide a reference to SRFI-10 (see References tree)}

In Common Lisp the possibility to interact with the reader is called a read macro or
reader macro, and this term will also be used throughout this thesis.

Emacs Lisp does not provide reader macros.  As of the upcoming version 25.1 the
reader is implemented in C and does not provide any hooks into it, apart from
the usual advice system, which all Emacs Lisp functions provide.

The code underlying this thesis provides a means to replace the reader built
into Emacs.  The main point of this thesis is to show that the flexible nature
of a Lisp environment may be used to replace even such a crucial part of the
system as the reader at runtime, without affecting unrelated parts, or making
the original part unavailable.
\end{abstract}

\section{Ideas}
\label{sec:orgheadline6}
\subsection{Basics: What does a reader do?}
\label{sec:orgheadline3}
\subsubsection{Read simple syntax (lists, symbols, strings, etc) (mention atoms?)}
\label{sec:orgheadline1}
\subsubsection{Read more complex structures (quote, function)}
\label{sec:orgheadline2}
\subsection{How do read macros work?}
\label{sec:orgheadline4}
\subsection{How do read macros look?}
\label{sec:orgheadline5}

\section{Introduction}
\label{sec:orgheadline7}
In this thesis, the author wishes to first convey how Lisps syntax is different
from other languages and why this is relevant to the discussion of extending the
syntax.  Secondly what is meant by an extensible syntax or read macro, and why
they might be relevant.  Also, depending on the background of the reader of this
text, one might confuse such syntactic extensions with Domain Specific
Languages—or DSLs for short.

While there is some overlap between the two, DSLs are not necessarily embedded,
but instead may come with a separate compiler and/or interpreter.  An extensible
syntax, as it is presented here, allows to change the syntax on-the-fly,
possibly without altering any present syntactic constructs, thus allowing new
constructs in an existing programming language.

\section{Motivation}
\label{sec:orgheadline8}
Users of other Lisp dialects, Common Lisp comes to mind, for a long time have
enjoyed the possibility to extend the syntax they use to write programs.  This
possibility provides a huge benefit in the course of evolving a language.
Macros---which are expanded at compile-time---provide a very powerful
mechanism.  These allow to control the order of evaluation of forms within the
macro, as well as the opportunity to transform them, yet sometimes even this
kind of power is not quite enough.

A very simple case, is the addition of syntax for data structures.  Most modern
programming languages provide some syntax to enter hash tables.  Take for
instance python, where the keys and values may be specified between curly
braces: 

\begin{minted}[]{python}
{"foo": "bar", "five": 5}
\end{minted}

When entered into a python interpreter, this expression returns a hash table (or
Dictionary in Python terminology).

Clojure, a rather recent addition to the Lisp family of languages also features
a similar syntax: 

\begin{minted}[]{clojure}
{"foo" "bar" "five" 5}
\end{minted}

This also returns a mapping from keys to values.

Unfortunately, neither Common Lisp nor Emacs Lisp feature such (concise) syntax\footnote{Emacs Lisp has read (and even print) syntax for hash tables, yet is is rather
ugly.  The same table looks like this: \#s(hash-table size 65 test eql
rehash-size 1.5 rehash-threshold 0.8 data ("foo" "bar" "five" 5))}.
Common Lisp users may create such a syntax themselves, yet Emacs Lisp users never had
this possibility.

One shortcut which is possible, is to simply define a function to do so;
consider the function \texttt{ht} from the source code of this thesis, ported
to Common Lisp:

\begin{minted}[]{common-lisp}
(defun ht (&rest args)
  "Create and return a hashtable.

Keys and values are given alternating in args."
  (let ((h (make-hash-table)))
    (loop for (key value) on args by #'cddr
       do (if (and key value)
              (setf (gethash key h) value)
              (error "Odd number of arguments passed")))
    h))
\end{minted}

This may be now used as:

\begin{minted}[]{common-lisp}
(ht "foo" "bar" "five" 5)
\end{minted}

instead of the more cumbersome

\begin{minted}[]{common-lisp}
(let ((h (make-hash-table)))
  (prog1 h
    (setf (gethash "foo" h) "bar")
    (setf (gethash "five" h) 5)))
\end{minted}

To further shorten and clarify the creation of hash tables, Common Lisp users may
use a read macro (assuming \texttt{ht} has been defined as above)\footnote{The
details of this code are not of particular importance, merely that it is
possible, and quite short.}:

\begin{minted}[]{common-lisp}
(set-macro-character #\} (get-macro-character #\)))

(set-macro-character
 #\{
 (lambda (stream char)
   (declare (ignore char))
   (apply #'ht (read-delimited-list #\} stream t))))
\end{minted}

Now the Common Lisp reader reads "\{\ldots{}\}" constructs in a similar fashion as the
Clojure reader does.

Given the code from this thesis, this code may be used in Emacs Lisp to produce
the same effect:

\begin{minted}[]{common-lisp}
(defun ht (&rest args)
  "Create and return a hashtable.

Keys and values are given alternating in args."
  (let ((h (make-hash-table)))
    (cl-loop for (key value) on args by #'cddr
             do (if (and key value) (puthash key value h)
                  (error "Odd number of arguments passed")))
    h))

(el-reader/set-macro-character ?\} (car (el-reader/get-macro-character ?\))))

(el-reader/set-macro-character
 ?\{
 (lambda (stream char)
   (apply #'ht (el-reader/read-delimited-list ?\} stream t))))
\end{minted}

Note the similarity of the API.  The prefixes have been put in place, as
Emacs Lisp provides no packaging facilities.\footnote{Should Emacs change the internal
representation of symbols, packaging may be worthwhile, given this reader.}

A more sophisticated use case might be to provide read syntax for regular
expressions---or regexps for short.  In Emacs Lisp regexps are notoriously ugly to
enter and read, because they are opaque strings, which are passed to a function,
which parses it at runtime.  This means that many constructs must be escaped
multiple times, which makes the strings difficult to read and write.  Here again
Clojure shows a nice way to handle the issue: by providing syntax for them.  In
Clojure, a regexp is entered by prefixing a double quoted string with a \#
character.

While Clojure provides more syntax than Common Lisp and Emacs Lisp do, it provides no
way for users to add their own shortcuts.

\section{Reader basics}
\label{sec:orgheadline17}
See also: \hyperref[sec:orgheadline8]{Motivation}
\subsection{Properties of Lisp Syntax}
\label{sec:orgheadline9}
The idea behind the syntax of Lisp is very different than in most other
programming languages.  While most languages provide constructs for assignment,
looping, function definition, etc, Lisp merely provides syntax for data
structures.  This idea has been reused in data description languages such as XML
and JSON, among others.  An expression in source form, which describes a Lisp
object is also called a symbolic expression, or sexp for short.  The term form
is also used on occasion.

So far, no evaluation rules have been discussed, and rightly so.  The evaluation
of code is not something the reader takes part in.  This is left to the
interpreter or the compiler.

\subsection{Intermezzo: syntax at runtime}
\label{sec:orgheadline10}
While some programming environments allow the user to load code at runtime, Lisp
also allows to read and compile code at runtime.  The former is exposed by a
function called \texttt{read}.  This function is given at least one argument:
something which may be used to retrieve a character, and to put back at least
one character which has been read.  This function attempts to read one Lisp
object from the character stream, and returns said object if successful.

This aspect may be compared to an XML parser, which also merely describes data.
One difference is that a Lisp reader can be more flexible, as this thesis shows.
A further difference is that standard Lisp syntax\footnote{Standard syntax refers to
the set of rules present when no modification has taken place since a fresh
start of the environment.} is shorter when a lot of markup is needed.
I.e. writing programs in XML tends to be rather verbose, as such languages as ant
\todo{Place some ref here.} show.  Writing a document in symbolic expressions
would probably be rather cumbersome, as either lots of strings must be used, or
elaborate rules for the treatment of special chars as apostrophes must be devised.

\subsection{Important Lisp Data Types}
\label{sec:orgheadline13}
A few Lisp data types are of particular importance, both because they are
relatively rare in other languages, but also because they are very widespread in
Lisp code, i.e. the reader often creates them.

Every Lisp dialect the author is aware of provides at least the following:

\subsubsection{Symbols}
\label{sec:orgheadline11}
The symbol is a type, which represents a name, often together with places for
values and/or functions.  The latter depend on the Lisp dialect.  An important
property of symbols is that fast lookup is supported, and that symbols are
interned.  This means that each symbol which is read, which has the same name,
returns the same Lisp object.\footnote{This is, in effect, a singleton.}

In Emacs Lisp, which is the Lisp under consideration for this thesis, a symbol has
the following attributes:

\begin{description}
\item[{Name}] A (typically hashed) name for the symbol.  This may be retrieved as a
string at any time.
\item[{Value}] This is the value of the symbol.  It may be retrieved by a call to
\texttt{symbol-value} or by simply appearing in a program (macros or
special forms may treat symbols differently).
\item[{Function}] This cell is looked up if the symbol is the first element of a
list which is evaluated (more on this later).  Alternatively, a call to
\texttt{symbol-function} returns the function to which it points (if
any). \todo{place a reference}
\item[{Property List}] A symbol may also have a Property List, or plist for short,
which is a mapping from a key (mostly a symbol) to a value.  This is
intended as a mechanism to store arbitrary metadata in a symbol.
\end{description}

\subsubsection{List}
\label{sec:orgheadline12}
In Lisp parlance, when a list is mentioned, it is almost always a singly linked
list.  While by far not the only compound data structure\footnote{Strictly speaking,
lists by themselves to not exist: there are only cons-cells, yet this nuance is
not very important for the current discussion.}, it is a very important one.

Not only is there syntax for lists---in the form of parentheses---together with
the symbol, the list is used as the prime representation for code.  The
following section hat some examples, as it discusses the result of calling the
reader.

\subsection{Reader, Compiler, Interpreter}
\label{sec:orgheadline16}
In this section, an overview over the reader, compiler and interpreter is given,
as it is relevant to the central point of the thesis.

\subsubsection{What does a reader do?}
\label{sec:orgheadline15}
In a sense, every programming language has a reader.  Every language
implementation must provide some means of converting characters in source code
into data structures which are more meaningful to the later stages of
interpretation or compilation.

The difference in Lisp is that the reader is available to the programmer both at
runtime and compile time (if there is a compile phase).

This means, that an application may store data structures by printing them (for
instance to a file) and later reading them back.  If one had a list, containing
the symbol \texttt{foo}, the number 3, the symbol \texttt{bar} and the
string ``4'', printing might result in this: \footnote{The exact form depends on the
dialect.  If not otherwise stated, Emacs Lisp is assumed.  Common Lisp is also used
extensively.}

\begin{minted}[]{common-lisp}
(foo 3 bar "4")
\end{minted}

This may later be read back with a call to \texttt{read}, to retrieve a data
structure, which has the same shape as the original.  Note that this is not
syntax for code in the sense of execution, yet merely for data.  This is an idea
which pervades Lisp, as code is data, and data may be code.

If the code which was read in this example were to be evaluated (by calling
\texttt{eval} on the result of the call to \texttt{read}), \texttt{foo}
would be expected to name a function, and \texttt{bar} would be evaluated.
The function named by \texttt{foo}, if any, would then be called with the
argument 3, whatever \texttt{bar} evaluated to, and the string ``4''.

The strict distinction between reading code, and evaluating code is what makes
reader macros possible.  It is also the reason for the parenthesis in typical
lisp code.  This is because the list (denoted by parenthesis) is the structure
in which code is held.  This is true for assignments, function calls, function
definitions etc.  Examples are given below.

\begin{enumerate}
\item Assignment
\label{sec:orgheadline14}
An assignment is either a macro (which expands to code), or a special form,
i.e. a primitive which the implementation must provide.  Here a special form was
chosen.  The value of the expression marked as \texttt{value} will be assigned
to the place noted by the symbol \texttt{variablename}.  In most cases, the
variable name must be quoted, i.e. be preserved as a symbol, instead of being
evaluated.

\begin{minted}[]{common-lisp}
(set 'variablename value)
\end{minted}
\end{enumerate}

\section{References}
\label{sec:orgheadline18}
\begin{description}
\item[{srfi-10}] \url{http://srfi.schemers.org/srfi-10/srfi-10.html}
\item[{dynamic scope misnomer}] \url{http://www.cs.cmu.edu/Groups/AI/html/cltl/clm/node43.html}
\end{description}
\end{document}