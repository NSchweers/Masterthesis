% Created 2016-03-18 Fri 14:40
\documentclass[b5paper,8pt,twoside]{article}
%\documentclass[a4paper]{article}
\usepackage[utf8]{inputenc}
\usepackage[T1]{fontenc}
\usepackage{fixltx2e}
\usepackage{graphicx}
\usepackage{longtable}
\usepackage{float}
\usepackage{wrapfig}
\usepackage{rotating}
\usepackage[normalem]{ulem}
\usepackage{amsmath}
\usepackage{textcomp}
\usepackage{marvosym}
\usepackage{wasysym}
\usepackage{amssymb}
\usepackage[hidelinks]{hyperref}
\usepackage{cite}
\tolerance=1000
\usepackage{minted}
\usepackage{todonotes}
\usepackage{comment}
\usepackage{listings}
\usepackage{lipsum}
%\usepackage{biblatex}
\author{Nathanael Schweers}
\date{\today}
\title{Proof of Concept: An implementation of a reader for extensible syntaxes for Emacs Lisp}
% \hypersetup{
%   pdfkeywords={},
%   pdfsubject={},
%   pdfcreator={Emacs 25.0.91.1 (Org mode 8.2.10)}}

\newcommand{\el}{Emacs Lisp}
\newcommand{\cl}{Common Lisp}
\newcommand{\elr}{el-reader}
\newcommand{\sym}[1]{\texttt{#1}}
\newcommand{\fun}[1]{\texttt{#1}}
\newcommand{\emacs}{GNU Emacs}

\newenvironment{myEnv} [2]
{\begin{frame} \frametitle{#1} \framesubtitle{#2} \begin{center}}
{\end{center} \end{frame}}

\newenvironment{sourcecode} [2]
{\begin{listing}[h] \label{#1} \caption{#2} \begin{verbatim}}
{\end{verbatim} \end{listing}}

\newminted{cl}{}

\begin{document}

\maketitle
\tableofcontents

\begin{abstract}
  This thesis discusses the design and implementation of a proof of concept for
  an \el{} compatible reader for extensible syntaxes, in \el{}, named
  ``\elr{}''.  In Lisp parlance, the reader is the part of the
  interpreter/compiler, which converts the characters in the source code to data
  structures, which are then further processed.  In an extensible reader, new
  syntactic constructs may be defined.  These new constructs take effect when
  specific characters are encountered in the source code, and may interact with
  the existing reader to create arbitrary data structures which are later either
  interpreted or compiled by the Lisp environment.

  Some other languages feature such capabilities, of which \cl{} is most widely
  known.  Some Scheme implementations also provide these or similar
  possibilities, yet this is not part of the standard.  SRFI-10 provides a
  specific character sequence with which all read macros must
  start. \cite{srfi-10}

  In \cl{} the possibility to interact with the reader is called a read macro or
  reader macro, and this term will also be used throughout this thesis.

  \el{} does not provide reader macros.  As of the upcoming version 25.1 the
  reader is implemented in C and does not provide any hooks into it, apart from
  the usual advice system, which all \el{} functions provide.

  The code underlying this thesis provides a means to replace the reader built
  into Emacs.  The main point of this thesis is to show that the flexible nature
  of a Lisp environment may be used to replace even such a crucial part of the
  system as the reader at runtime, without affecting unrelated parts, or making
  the original part unavailable.
\end{abstract}

\section{Ideas}
\label{sec-1}

\begin{listing}[h]
\label{lbl:test}
\caption{A caption}
\begin{verbatim}
(message "hello")
\end{verbatim}
\end{listing}

% \begin{sourcecode}{mint:test}{A caption}
% (message "Hello")
% \end{sourcecode}

\subsection{Basics: What does a reader do?}
\label{sec-1-1}
\subsubsection{Read simple syntax (lists, symbols, strings, etc) (mention atoms?)}
\label{sec-1-1-1}
\subsubsection{Read more complex structures (quote, function)}
\label{sec-1-1-2}
\subsection{How do read macros work?}
\label{sec-1-2}
\subsection{How do read macros look?}
\label{sec-1-3}

\section{Introduction}
\label{sec:introduction}
In this thesis, a \cl{} like extensible reader for \el{} is discussed.  Before
discussing the reader itself, it is important to discuss what a reader does,
what it can do, and how the built-in reader in \emacs{} differs from the
extensible one discussed in this thesis, or the \cl{} reader, from which this
reader draws its inspiration.

For this purpose, Lisps syntax, and how it is different from almost all other
languages, is discussed.

It is also shown what can be done with such an extensible reader, and why one
might want to use such extensions.

% The application of such an extensible reader allows the creation of new
% syntactic constructs, which may be embedded into the language.  Thus, it enables
% the user to not only create a DSL, but also to embed it into a greater program,
% without making the new constructs opaque by putting them into strings, which
% must be parsed at runtime.


% In this thesis, the author wishes to first convey how Lisps syntax is different
% from other languages and why this is relevant to the discussion of extending the
% syntax.  Secondly what is meant by an extensible syntax or read macro, and why
% they might be relevant.  Also, depending on the background of the reader of this
% text, one might confuse such syntactic extensions with Domain Specific
% Languages—or DSLs for short.

% While there is some overlap between the two, DSLs are not necessarily embedded,
% but instead may come with a separate compiler and/or interpreter.  An extensible
% syntax, as it is presented here, allows to change the syntax on-the-fly,
% possibly without altering any present syntactic constructs, thus allowing new
% constructs in an existing programming language.

\section{Terminology}
\label{sec:terminology}

To get things out of the way, we first define some terms:
\begin{description}
\item[Terminating macro character] A character which has been marked as a
  terminating macro character must have a function attached to it, even if this
  function only signals an error when called.  Such characters also terminate
  any token which occurs before them.  If, for instance, `(' is such a
  character, the character sequence ``bar(...'' will stop reading between the
  `r' and `(', resulting in the symbol \sym{bar}.

  If such a character is encountered as the first char to \fun{read}, its
  associated function is called.
\item[Non-terminating macro character] These characters work in a similar
  fashion to Terminating macro characters, yet do not terminate a token
  preceding them.  Also their associated function is not called in such a
  situation.
\item [Read macro] A read macro is a pair of a (non-)terminating macro character
  and a function, which is associated with it.  The two together create the
  possibility to define new syntactic constructs.
\item[List] In Lisp parlance, a list almost always means a singly linked list,
  i.e. a collection which can grow and shrink at the front by one element in
  constant time.
\end{description}

\section{Motivation}
\label{sec:motivation}
Users of other Lisp dialects, especially of \cl{}, have enjoyed the possibility
to modify the syntax they use to write programs for a long time.  This
possibility provides a huge benefit in the course of evolving a language, as
well as implementing readers for custom languages.  Macros---which are expanded
at compile-time---provide a very powerful mechanism for code generation.  These
allow to control the order of evaluation of forms within the macro, as well as
the opportunity to transform them, yet sometimes even this kind of power is not
quite enough, as the macro call must still consist of valid s-expressions.

A very simple case is the addition of syntax for data structures, since this is
what the reader is there for anyway: transforming characters into data
structures.  Most modern programming languages provide some syntax to enter hash
tables.  Take for instance python, where the keys and values may be specified
between curly braces:

\begin{minted}[]{python}
{"foo": "bar", "five": 5}
\end{minted}

When entered into a python interpreter, this expression returns a hash table (or
Dictionary in Python terminology).

Clojure, a rather recent addition to the Lisp family of languages also features
a similar syntax:

\begin{minted}[]{clojure}
{"foo" "bar" "five" 5}
\end{minted}

This also returns a mapping from keys to values.

Unfortunately, neither \cl{} nor \el{} feature such (concise)
syntax\footnote{\el{} has read (and even print) syntax for hash tables, yet is
  is rather ugly.  The same table looks like this: \#s(hash-table size 65 test
  eql rehash-size 1.5 rehash-threshold 0.8 data ("foo" "bar" "five" 5))}.  \cl{}
users may create such a syntax themselves, yet \el{} users never had this
possibility.

In the absense of read macros, one possible shortcut is to simply define a
function to do so; consider the function \fun{ht} in listing \ref{mint:cl-ht}
from the source code of this thesis, ported to \cl{}. In \ref{lbl:test}

\lipsum[0]

\lstset{language=Lisp}
\begin{lstlisting}[caption=my caption\label{lbl:test},frame=single,float]
(defun ht (&rest args)
  "Create and return a hashtable.

Keys and values are given alternating in args."
  (let ((h (make-hash-table)))
    (loop for (key value) on args by #'cddr
       do (if (and key value)
              (setf (gethash key h) value)
              (error "Odd number of arguments passed")))
    h))
\end{lstlisting}


\begin{listing}[h]
\caption{Creating a hash table.}
\label{mint:cl-ht}
\begin{clcode}
(defun ht (&rest args)
  "Create and return a hashtable.

Keys and values are given alternating in args."
  (let ((h (make-hash-table)))
    (loop for (key value) on args by #'cddr
       do (if (and key value)
              (setf (gethash key h) value)
              (error "Odd number of arguments passed")))
    h))
\end{clcode}
\end{listing}
This may be used as

\begin{clcode}
(ht "foo" "bar" "five" 5)
\end{clcode}

instead of the more cumbersome and repetitive code shown in listing
\ref{mint:cl-canon-ht}.

\begin{listing}[h]
\begin{clcode}
(let ((h (make-hash-table)))
  (prog1 h
    (setf (gethash "foo" h) "bar")
    (setf (gethash "five" h) 5)))
\end{clcode}
\caption{Canonical way to create a hash table.}
\label{mint:cl-canon-ht}
\end{listing}

To further shorten and clarify the creation of hash tables, \cl{} users may use
a read macro. Listing \ref{mint:cl-hash-reader} shows how this may be done
(assuming \fun{ht} has been defined as in listing \ref{mint:cl-ht}).
\footnote{The details of this code are not of particular importance, merely that
  it is possible, and quite short.}

\begin{listing}[h]
\begin{clcode}
(set-macro-character #\} (get-macro-character #\)))

(set-macro-character
 #\{
 (lambda (stream char)
   (declare (ignore char))
   (let ((k-v (read-delimited-list #\} stream t)))
     (if (oddp (length k-v))
         (error "Invalid syntax: {}")
         `(ht ,@k-v)))))
\end{clcode}
\caption{A \cl{} read macro to read hash tables.}
\label{mint:cl-hash-reader}
\end{listing}

Now the \cl{} reader reads ``\{\ldots{}\}'' constructs in a similar fashion as
the Clojure reader does.

Given \elr{}, i.e. the code from this thesis, the code in listing
\ref{mint:el-hash-reader} may be used in \el{} to produce the same effect.

\begin{listing}[h]
\begin{clcode}
(defun ht (&rest args)
  "Create and return a hashtable.

Keys and values are given alternating in args."
  (let ((h (make-hash-table)))
    (cl-loop for (key value) on args by #'cddr
             do (if (and key value) (puthash key value h)
                  (error "Odd number of arguments passed")))
    h))

(el-reader/set-macro-character
 ?\} (car (el-reader/get-macro-character ?\))))

(el-reader/set-macro-character
 ?\{
 (lambda (stream _char)
   (cl-values
    (let ((k-v (el-reader/read-delimited-list ?\} stream t)))
      (if (= (mod (length k-v) 2) 1)
          (error "Invalid syntax: {}")
        `(el-reader//ht ,@k-v))))))
\end{clcode}
\label{mint:el-hash-reader}
\caption{An equivalent read macro using \elr{}.}
\end{listing}

Note the similarity of the API.  The prefixes have been put in place, as \el{}
provides no packaging facilities.\footnote{Should Emacs change the internal
  representation of symbols, packaging may be possible, given this reader, yet
  is beyond the scope of this thesis.}

A more sophisticated use case might be to provide read syntax for regular
expressions---regexps for short.  In \el{} regexps are notoriously ugly to enter
and read, because they are opaque strings, which are passed to a function, which
parses it at runtime.  This means that many constructs must be escaped multiple
times, which makes the strings difficult to read and write.  A particular
notorious offender is the regular expression which matches a single backslash;
in Clojure one may write \texttt{\#"\textbackslash\textbackslash"}, \el{} users
are forced to write the much more cumbersome
\texttt{"\textbackslash\textbackslash\textbackslash\textbackslash"}.  Here again
Clojure shows a nice way to handle the issue: by providing syntax for them.

While Clojure provides more syntax than \el{} does, and more than \cl{} does out
of the box, it provides no way for users to add their own constructs.

Finally, it is also possible to define non-standard evaluation semantics, as the
code for read macros is run at read time.  This means that calculations may be
performed at read time, which---when using ahead of time
compilation---translates to compile-time.

An example for this is something similar to rubys \texttt{\%w} construct.  It
reads a list of tokens, converts them to strings and makes an array out of
them.  

\begin{minted}{ruby}
%w(foo bar spam and eggs) => ["foo", "bar", "spam", "and", "eggs"]
\end{minted}

While ruby is very flexible about the delimiters, one might define
\texttt{\#w(...)} in Lisp as in listing \ref{mint:elr-ruby-words}

\begin{listing}[h]
\begin{clcode}
(defvar *my-read-stuff/word-seq-types*
  {'cons #'list 
   'vector #'vector})

(defun my-read-stuff/read-words (stream char _num)
   (let ((words (el-reader/read stream t nil t)))
     (if (not (seqp words))
         (error "Invalid syntax: #w")
       (cl-values
        `(quote
          ,(apply
            (gethash (type-of words)
                     *my-read-stuff/word-seq-types*)
            (seq-map (lambda (s)
                       (if (not (symbolp s))
                           (error "Invalid syntax: #w")
                         (symbol-name s)))
                     words)))))))

(el-reader/set-dispatch-macro-character
 ?# ?w
 #'my-read-stuff/read-words)
\end{clcode}
\label{mint:elr-ruby-words}
\caption{A read macro for rubys words syntax.}
\end{listing}

While this version uses different delimiters for different types of return value
(parentheses form a list, square brackets form a vector), it can only use
certain delimiters, namely those which have read syntax for a sequential data
structure.  On the other hand it is extensible in the types it provides, and
most importantly forms the structure at read-time.  This may be of some worth
for large constants.

Here is how it is used:

\begin{clcode}
#w(foo bar spam eggs) => ("foo" "bar" "spam" "eggs")
#w[foo bar spam eggs] => ["foo" "bar" "spam" "eggs"]
\end{clcode}

\section{Lisp Basics}
\label{subsec:lisp-basics}

This section is only intended as a short recap of what Lisp syntax is, and how
it relates to the reader.  For more details, a Lisp textbook is more
appropriate.  A good reference for \el{} is found at \cite{elisp-reference}.
The chapters 

Lisp syntax never directly describes control flow, function definitions, or
other actions, but merely data.  If the described data consists of lists and
symbols, it will later on be treated as code, which can be compiled and
executed.\todo{Possibly show some Lisp examples.}

Such data structures can represent code by having a list with an ``action'',
which may be a special form, a macro or a function, as its first element.  Any
further elements are arguments to said operator.

As an example, here is some code which performs a variable binding with
\fun{let} (which is a special form), an assignment with \fun{setf} (a macro) and
a function call to \fun{message}.

\begin{clcode}
(let ((var 5))
  (setf var "Hello")
  (message "%s" var))
\end{clcode}

When Lisp source code is processed, it goes through three distinct stages:
reading, macro expansion (optionally this can also be a compile phase), and
evaluation (which can be the execution of compiled code).

The first stage is somewhat unique to Lisp---most languages only have this as an
implementation detail in the compiler or interpreter.

\subsection{Reader}
\label{subsec:reader}

The reader is the first stage which code runs through.  This part of the
evaluation mechanism is the only part which sees individual characters.  As Lisp
syntax consists of literal representation of data, not of control structures and
the like, the reader transforms these literal representations into data
structures, which may be treated as executable code later on.

The reader is the part of the evaluator which knows that parentheses denote
lists, square brackets denote vectors, how to read numbers, symbols and so on.

It also defines where a token ends, in the absence of whitespace. For instance,
the following expressions are all slightly different textual representations of
the same object, i.e. later stages cannot distinguish between the two.

\subsection{Macro expansion}
\label{subsec:macro-exp}

\subsection{Evaluator}
\label{subsec:evaluator}



\section{Reader basics}
\label{sec:reader-basics}

Before discussing what the reader is and what it does, it is important to first
introduce a few important data types.
\subsection{Important Lisp Data Types}
\label{subsec:important-types}

A few Lisp data types are of particular importance, both because they are
relatively rare in other languages, but also because they are very widespread in
Lisp code, i.e. the reader often creates them.

Every Lisp dialect the author is aware of provides at least the following:

\subsubsection{Symbols}
\label{subsubsec:symbols}

The symbol is a type, which represents a name, often together with places for
values and/or functions.  The latter depend on the Lisp dialect.  An important
property of symbols is that fast lookup is supported, and that symbols are
interned.  This means that each symbol which is read, which has the same name,
returns the same Lisp object.\footnote{This is, in effect, a singleton.}

In \el{}, which is the Lisp under consideration for this thesis, a symbol has
the following attributes:

\begin{description}
\item[{Name}] A (typically hashed) name for the symbol.  This may be retrieved
  as a string at any time.
\item[{Value}] This is the value of the symbol.  It may be retrieved by a call
  to \fun{symbol-value} or by simply appearing in a program (macros or special
  forms may treat symbols differently).
\item[{Function}] This cell is looked up if the symbol is the first element of a
  list which is evaluated (more on this later).  Alternatively, a call to
  \fun{symbol-function} returns the function to which it points (if any).
\item[{Property List}] A symbol may also have a Property List, or plist for
  short, which is a mapping from a key (mostly a symbol) to a value.  This is
  intended as a mechanism to store arbitrary metadata in a symbol.
\end{description}

\subsubsection{List}
\label{subsubsec:list}

In Lisp parlance, when a list is mentioned, it is almost always a singly linked
list.  While by far not the only compound data structure\footnote{Strictly
  speaking, lists by themselves to not exist: there are only cons-cells, yet
  this nuance is not very important for the current discussion.}, it is a very
important one.

Not only is there syntax for lists---in the form of parentheses---together with
the symbol, the list is used as the prime representation for code.  The
following section hat some examples, as it discusses the result of calling the
reader.
\subsection{Lisp basics}
\label{subsec:lisp-basics}

This section is about a few basics of Lisp, not so much the reader.  Here it is
discussed how nested lists can represent such actions as function calls,
assignment and function definitions, as Lisp does not have any syntax for those.
\paragraph{Function call}
\label{par:function-call}

In Lisp a function call happens when the evaluator comes across a list form.  In
such a case the first element is taken to be a function and the other elements
are its arguments.  Here is the canonical hello world program for \el{}:

\begin{clcode}
(print "Hello, world")
\end{clcode}

Here we have a list of two elements.  The first element happens to be a symbol,
which has a function associated with it.  The second element (a string in this
case) is evaluated (nothing special happens here, as strings evaluate to
themselves, contrary to lists) and passed as the sole argument to the function
named by the symbol print.

\paragraph{Assignment}
\label{par:assignment}

An assignment is either a special form, i.e. a primitive which the
implementation must provide, or a macro (which expands to other code, eventually
leading to a special form).  Here the low level special form was chosen.  To
assign the value 5 to the ``variable'' \sym{x} with a such a low level special
form, we could write the following:

\begin{clcode}
(set (quote x) 5)
\end{clcode}

The \fun{quote} form around \sym{x} tells the evaluator to \emph{not} evaluate
\sym{x}, but to pass it on as the symbol itself.  Note that the reader would
have created the symbol \sym{x}, and only the evaluator can care about any
meaning for \sym{x}.  Many language implementations throw symbols away at
runtime.  This is also possible in Lisp, as the compiler can remove symbols and
their lookups in many situations.

Note that although \fun{set} is a special form, it too is invoked when it is the
first element in a list.  Like macros, but unlike functions, special forms may
control how to evaluate their arguments.

\paragraph{Function definition}
\label{par:function-def}
Unsurprisingly, a function is defined by a list form, with a special symbol as
its first element.  In \el{} this symbol is \fun{defun}.

\begin{clcode}
(defun say-hello (text)
  (print text))
\end{clcode}

Note that the third element, although it is a list, is neither evaluated, nor
must it be quoted.  This is because \fun{defun} is a macro, and macros may
control how to evaluate their arguments.

\subsection{What does the reader do?}
\label{subsec:what-does-the-reader-do}
In a sense, every programming language has a reader.  Every language
implementation must provide some means of converting characters in source code
into data structures which are more meaningful to the later stages of
interpretation or compilation.

The difference in Lisp is that the reader is available to the programmer both at
runtime and compile time (if there is a compile phase).

This means, that an application may store data structures by printing them (for
instance to a file) and later reading them back.  If one had a list, containing
the symbol \sym{foo}, the number 3, the symbol \sym{bar} and the string "4",
printing might result in this: \footnote{The exact form depends on the dialect.
  If not otherwise stated, is assumed.  is also used extensively.}

\begin{clcode}
(foo 3 bar "4")
\end{clcode}

This may later be read back with a call to \fun{read}, to retrieve a data
structure, which has the same shape as the original.  Note that this is not
syntax for code in the sense of execution, yet merely for data.  This is an idea
which pervades Lisp, as code is data, and data may be code.

If the code which was read in this example were to be evaluated (by calling
\fun{eval} on the result of the call to \fun{read}), \sym{foo} would be expected
to name a function, and \sym{bar} would be evaluated as a variable.  The
function named by \sym{foo}, if any, would then be called with the argument 3,
whatever \sym{bar} evaluated to, and the string "4".

The strict distinction between \emph{reading}, and \emph{evaluating} code is
what makes read macros possible.  It is also the reason for the parenthesis in
typical lisp code.  This is because the list (denoted by parenthesis) is the
structure in which code is held.  This is true for assignments, function calls,
function definitions etc.  Examples are given below.

\subsection{Properties of Lisp Syntax}
\label{subsec:properties-of-lisp-syntax}

The idea behind the syntax of Lisp is very different than in most other
programming languages.  While most languages provide constructs for assignment,
looping, function definition, etc, Lisp merely provides syntax for data
structures.  This idea has been reused in data description languages such as XML
and JSON, among others.  An expression in source form, which describes a Lisp
object is also called a symbolic expression, or sexp for short.  The term form
is also used on occasion.

So far, no evaluation rules have been discussed, and rightly so.  The evaluation
of code is not something the reader takes part in.  This is left to the
interpreter or the compiler.

\subsection{Intermezzo: syntax at runtime}
\label{subsec:intermezzo:syntax-at-runtime}

While some programming environments allow the user to load code at runtime, Lisp
also allows to read and compile code at runtime.  The former is exposed by a
function called \fun{read}.  This function is given at least one argument:
something which may be used to retrieve a character, and to put back at least
one character which has been read.  This function attempts to read one Lisp
object from the character stream, and returns said object if successful.

This aspect may be compared to an XML parser, which also merely describes data.
One difference is that a Lisp reader can be more flexible, as this thesis shows.
A further difference is that standard Lisp syntax\footnote{Standard syntax
  refers to the set of rules present when no modification has taken place since
  a fresh start of the environment.} is shorter when a lot of markup is needed.
I.e. writing programs in XML tends to be rather verbose, as such languages as
ant show \cite{ant}.  Writing a document in symbolic expressions would probably
be rather cumbersome, as either lots of strings must be used, or elaborate rules
for the treatment of special chars as apostrophes must be devised.

\subsection{Reader, Compiler, Interpreter}
\label{subsec:reader-compiler-interpreter}

In this section, an overview over the reader, compiler and interpreter, together
with their interactions is given, as it is relevant to the central point of the
thesis.

\section{Why Emacs?}
\label{sec:why-emacs}

The author chose to extend Emacs with an extensible reader, and not some other
member of the Lisp family, such as Clojure for several reasons.  One reason is
that the author has gained the most familiarity with \el{} and \cl{}.  The two
are reasonably similar in a number of ways, especially with \el{} growing a lot
in recent years.  \el{} used to have only dynamic scope, yet gained optional
lexical scope in version 24.1\cite{Emacs-Lexical},% \footnote{See
  % \url{https://www.gnu.org/software/emacs/news/NEWS.24.1}}
giving \el{} both lexical and dynamic scope, as with \cl{}.  Also, in the next
version of Emacs, which will probably become version
25.1\cite{emacs-pretest},% \footnote{See
  % \url{http://permalink.gmane.org/gmane.emacs.announce/59} for a source tarball.
  % It contains a file etc/NEWS which details changes}
CLOS-style multiple dispatch generic functions have been added, which ease
development for many scenarios and again are a step towards \cl{}.

\el{} is also quite similar to \cl{} in many other rather minor respects.  Both
are a Lisp-2\footnote{Being a Lisp-2 means having separate namespaces for
  functions and variables.}, both use \sym{defun} to define functions and
\sym{defmacro} to define macros, both with a very similar and largely compatible
grammar.  \el{} also sports a built-in package called CL which tries to emulate
many \cl{} features, such as more features for default arguments, destructuring,
and the \cl{} \fun{loop} macro.

Emacs still has a long way to go before being as powerful as \cl{}, and probably
never will be, as multithreading, conditions and restarts, full CLOS support,
packages and some minor issues like returning multiple values and global symbol
macros are still missing.

As Emacs also does not feature a built-in extensible reader, this thesis
provided one to hopefully drive Emacs development forward some more.

\section{Reader internals}
\label{sec:reader-internals}

While the reader has a very simple interface, internally it must keep some
state, in order to know how to treat the characters encountered.  Some ephemeral
state is kept on the call stack and in closures, but the more important state is
kept in a structure called a readtable.

The readtable enables users to modify this state.  While it is possible for a
user to modify this structure directly, it is strongly discouraged.  Emacs has
no possibility to prevent users from seeing or modifying state, but through
compiled closures.

As a convention, symbols which are part of the public API are prefixed with
``el-reader/'', while symbols which are not part of the public API are prefixed
with ``el-reader//''.  Some symbols also use a sort of sub-namespace, i.e. are
prefixed with ``el-reader//rt/''.  This helps users to distinguish between
public API and internals on the basis of a symbol name.

The public API consists mostly of functions like
\fun{el-reader/set-macro-character}, which manipulate a readtable.

Many of these functions take a readtable as an optional argument.  If no
argument is given, the current readtable is used.  The current readtable is a
special variable named \sym{*el-reader/readtable*}.  The default value of this
variable is a readtable which mimics \el{}s default syntax as closely as
possible.  While this has not been achieved at the moment, it is possible to
complete the table with more read macros.  The heavy lifting, such as reading
circular data structures is already present.  Also, the most common constructs
used in \el{} are present.  

The readtable provides the basis on which \elr{} decides how to treat
characters.  Characters have exactly one syntax type.

Note that not all characters which denote the same value have the same syntax
type and traits.  It is possible for a character (for instance when escaped) to
be treated as a constituent with the alphabetic trait, although it would
otherwise denote a macro character.  For instance: `(' is a macro character, but
if read as ``$\backslash$('', the open parenthesis will instead be treated as an
alphabetic constituent.  Traits such as alphabetic will be discussed in section
\ref{subsubsec:traits}.

\subsection{Syntax types}
\label{subsec:syntax-types}

Here all syntax types are listed.  These are very similar to the syntax types in
\cl{}, yet differ in a few points, as the main goal was not to parse \cl{} code,
but to be compatible to \el{}.

\subsubsection{Constituent}
\label{par:constituent}

Most characters are treated as ``constituent'' characters.  To be precise: if a
character does not have a different syntax type, it is constituent.  Constituent
characters may be part of a token.

Constituent characters also have traits, which will be discussed in the next
section.
\subsubsection{Invalid}
\label{subsubsection:invalid}

If the reader encounters an invalid character, an error is raised.  This does
not imply that no such char may be in the input stream.  A read macro may choose
to skip such characters, or may use them in creative ways.  This is completely
up to the user.  The default readtable does not specify any characters to have
the invalid syntax type.

\subsubsection{Whitespace}
\label{par:whitespace}

These are characters which serve no special meaning.  The name should be pretty
explanatory.

\subsubsection{Single escape}
\label{par:single-escape}

In standard \el{} a single escape char allows spaces and other characters which
do not comprise a symbol by itself to be part of one.  The sequence "foo bar"
would normally denote two tokens (from which two symbols will be created).
Should the desired name of \emph{one} symbol be "foo bar", a backslash is needed
(which is the sole default single escape char): "foo$\backslash$ bar" would read
as one token, not two.  This works in vanilla \el{}, although the author has
never seen this in the wild.

In \cl{} this is more important, as the reader up-cases all chars by default,
which escaping also inhibits.  This aspect of \cl{} was not incorporated into
\elr{}, as it is probably only present in \cl{} for historic reasons.

\subsubsection{Multiple escape}
\label{par:multiple-escape}

These chars are similar to single escape chars.  While a single escape only
escapes the next char, a multiple escape char escapes all chars up to the next
multiple escape char.  In \cl{} the `|' character is used by default, in
\elr{} the functionality is present, yet no character has been assigned, as
most users would not expect the pipe character to behave in such a way.

\subsubsection{Terminating macro character}
\label{par:terminating-macro-char}

These characters are what make read macros so interesting.  Every macro
character has an associated function, which is called with the character and the
current stream as arguments.

A function associated with a macro char may have no side effects apart from
advancing the stream, as such a function may make no assumptions about the
number of times an object is read.  Such a function may read one char at a time,
put back one read char which it has just read, and call \fun{read} on the
(possibly advanced) stream.

The function which is associated with a macro character shall return the read
object.

Terminating macro characters also terminate any token which may have been read
before encountering the macro character.  For example, `(' is a Terminating
macro character, which reads a list.  When the reader encounters the string
"foo(\ldots{}" it reads the constituents `f' `o' and `o'.  When `('---which is a
\emph{terminating} macro character---is encountered, it is put back into the
stream, and "foo" is treated as a token.  In the next call to \fun{read} (should
there be any), the function associated with `(' will be called, which in turn is
the result of the read call.

\subsubsection{Non-terminating macro character}
\label{subsubsec:non-terminating-macro-char}

These characters are very similar to \emph{terminating} macro characters.  The
only difference is how such a character is treated when constituents have been
read.  While reading a token, a terminating macro character is put back into the
stream and the already read characters form a token.  When a
\emph{non}-terminating macro character is encountered, the character is treated
as if it was a constituent and its function is \emph{not} called.  If no
constituents have been read before, an non-terminating macro character is
treated the same as a terminating macro character.


\subsubsection{Traits}
\label{subsubsec:traits}

Every character which is a constituent also has one or more traits, which
provide further information on its use.

\paragraph{Alphabetic}
\label{par:alphabetic}

characters are characters which may appear in a token.  Escaping may force this
trait on an otherwise non-constituent character.

\paragraph{Digit}
\label{par:digit}

characters may be treated as a number, if they form a token, and they are
compatible with the input base.

\paragraph{Package marker}
\label{par:package-marker}

characters were only implemented because \cl{} has them.  The only character
with this trait is the colon.  As Emacs uses so called obarrays to intern its
symbols, and it is not possible to intern a symbol into more than one array
%\cite{no-multi-intern}
, it is not possible to clone \cl{}s packaging mechanism to Emacs.  Hence this
syntax type is not of particular importance.

\paragraph{Plus sign, minus sign, dot, decimal point, ratio marker and exponent marker}
\label{par:number-traits}

are used while constructing numbers.

\paragraph{Invalid}
\label{par:invalid}

Should a character with this trait be encountered, an error is thrown.  So far,
\elr{} does not specify any characters to be of invalid trait.

\section{API overview}
\label{sec:api-overview}

The API of \elr{} takes lots of inspiration from the corresponding API in
\cl{}, as the two languages are sufficiently similar, and \cl{} has a very
mature library.  Also, it may be expected that some \el{} developers are
familiar with \cl{} read macros.  Those who are not may benefit from the
similarity, as learning and reference material on \cl{} read macros seems to be
more widely available than for any other Lisp, probably due to its stable
standard.  \footnote{Paul Grahams book ``On Lisp''\cite[p.~224]{on-lisp}
  contains a chapter explaining both regular macros and read macros.}

\subsection{The readtable}
\label{subsec:readtable}

The only user visible data structure is the readtable.  In \cl{} this type has
non-enumerable types, while in \elr{} it is defined in terms of
\fun{defclass}, hence is enumerable and even has public attributes.  Most users
will need none of these.  They are mostly needed when creating a completely new
readtable from scratch.

\subsection{Functions}
\label{subsec:functions}

Here be dragons.

% \section{References}
% \label{sec:references}
% \begin{description}
% \item[{srfi-10}] \url{http://srfi.schemers.org/srfi-10/srfi-10.html}
% \item[{dynamic scope misnomer}] \url{http://www.cs.cmu.edu/Groups/AI/html/cltl/clm/node43.html}
% \end{description}

\bibliographystyle{abbrvdin}
\bibliography{references}

\end{document}

%%%% Local Variables:
%%%% TeX-command-extra-options: "-shell-escape"
%%%% End: